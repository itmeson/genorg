\documentclass[addpoints, 12pt]{exam}
\pagestyle{headandfoot}
\firstpageheader{Name:}{Exam I}{June 7, 2012}
\firstpageheadrule
\pointsinmargin

\usepackage{lewis}

\begin{document}
\begin{questions}

\question[8]
Classify each of the following changes as a physical property,
chemical property, physical change, or chemical change:
\begin{parts}
  \part A substance is reacted with chlorine gas
 
  \part A substance will explode if exposed to light

  \part A substance has a high density

  \part A substance has a freezing point of $-20^{\circ} \textbf{C}$

  \part A substance is melted
\end{parts}


\question[8] Identify the elements and the number of atoms of each type for
  the following formulas:
\begin{parts}
  \part  $\textbf{Fe}_2\textbf{S}_3$

  \part  $\textbf{CaCl}_2$

  \part $\textbf{H}_2\textbf{SO}_4$

  \part $\textbf{KF} $
\end{parts}


\question[4] Choose all that apply:  An element is a substance which
\begin{choices}
  \choice can be broken down into simpler substances by physical means

  \choice cannot be broken down into simpler substances by physical means

  \choice   can be broken down into simpler substances by chemical means

  \choice  cannot be broken down into simpler substances by chemcal means
  
\end{choices}

\question[4] Which of the following is a property of both gases and
liquids?
\begin{choices}
  \choice definite shape
 
 \choice indefinite shape

 \choice definite volume

 \choice indefinite volume
\end{choices}

\question[4] How many electrons appear in the Lewis dot symbol for an
element whose electron configuration is $1s^22s^22p^3$?

\begin{choices}
\choice 2

\choice 3

\choice 5

\choice 7

\choice 8
\end{choices}


\question[5]
If the density of a solution is $3.2 \frac{grams}{mL}$, how many mL would it take to get 100 grams?

\question[10]
Fill in the following table of names and symbols of elements:

\begin{tabular}{| l | l |}
\hline
Symbol & Name \\
\hline
Au & \\
Pb & \\
 & Sodium\\
 & Hydrogen\\
Hg & \\
Li & \\
Be & \\
 & Nitrogen \\
 & Chlorine \\
P & \\
\hline
\end{tabular}

\question[10]
Fill in the following table:

\begin{tabular}{| c | c | c | c | c | c |}
\hline
Symbol & Atomic & Mass & Number of&Number of&Number of \\
 & Number & Number & Protons & Neutrons & Electrons \\
\hline
$^{23}\textbf{V}$ & & & & & \\
\hline
 & 55 & 133 & & & 55\\
\hline
& & 77 & 33 & & 33\\ 
\hline
$\textbf{Br}^{1-}$ & & & & & \\
\hline
& & & 20 & 20 & 18 \\ 
\hline
\end{tabular}

\question[10]  Write chemical formulas for the compounds formed between the following positive and negative ions.
\begin{parts}
\part  $\textbf{Al}^{3+}$ and $\textbf{CO}_3^{2-}$ 

\part $\textbf{Fe}^{3+} $ and $\textbf{OH}^-$  
\end{parts}


\question[10]
The following Lewis diagrams are for two unspecified elements shown in
their \emph{neutral form}.  How many atoms of each type would be
required to form a neutral \emph{compound}?  What is the charge on
each ion type? 

\begin{parts}
\part \lewis{X}{.}{.}{.}{.}{.}{.}{.}{}  - \lewis{Y}{}{}{.}{}{}{}{}{}

\part \lewis{X}{.}{.}{.}{.}{}{}{.}{.} - \lewis{Y}{.}{}{.}{}{.}{}{}{}
\end{parts}


\question[10]  Fill in a shell diagram for a Phosphorus (\textbf{P}) atom.
  \begin{parts}
    \part How many shells (n-levels) does it need?

    \part How many electrons fit in each of the inner (core) shells?

    \part How many electrons are in the outer (valence) shell?

    \part Fill in the diagram.

    \part If \textbf{P} had a full outer shell, what would the overall charge be?
 \end{parts}

\question[5] Match up one entry from the left column with an entry from the other
  column that has similar chemical properties, and, asssuming that
  these represent \emph{neutral} atoms, what elements do they each represent?

\begin{tabular}{|l|l|l|l|}
a& $1s^22s^22p^6$ & 2& $1s^22s^22p^63s^2$ \\
b& $1s^22s^1$ & 1& $1s^22s^22p^63s^23p^6$ \\
c& $1s^22s^2$ & 3& $1s^22s^22p^63s^23p^3$ \\
d& $1s^22s^22p^3$ & 4&$1s^22s^22p^63s^1$  
\end{tabular}

\question[5]  Explain \emph{why} the elements in [12] have similar
chemical properties.




\end{questions}

\end{document}
