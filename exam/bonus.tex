\documentclass[addpoints, 12pt]{exam}
\pagestyle{headandfoot}
\firstpageheader{Name:}{Final Exam Bonus Section}{June 28, 2012}
\firstpageheadrule
\pointsinmargin
\usepackage{mhchem}
\usepackage{lewis}

\begin{document}
\begin{questions}
\section{}

\question[8]
Classify each of the following changes as a physical property,
chemical property, physical change, or chemical change:
\begin{parts}
  \part A substance is reacted with chlorine gas
 
  \part A substance will explode if exposed to light

  \part A substance has a high density

  \part A substance has a freezing point of $-20^{\circ} \textbf{C}$

  \part A substance is melted
\end{parts}


\question[8] Identify the elements and the number of atoms of each type for
  the following formulas:
\begin{parts}
  \part  $\textbf{Fe}_2\textbf{S}_3$

  \part  $\textbf{CaCl}_2$

  \part $\textbf{H}_2\textbf{SO}_4$

  \part $\textbf{KF} $
\end{parts}


\question[4] Choose all that apply:  An element is a substance which
\begin{choices}
  \choice can be broken down into simpler substances by physical means

  \choice cannot be broken down into simpler substances by physical means

  \choice   can be broken down into simpler substances by chemical means

  \choice  cannot be broken down into simpler substances by chemcal means
  
\end{choices}

\question[4] Which of the following is a property of both gases and
liquids?
\begin{choices}
  \choice definite shape
 
 \choice indefinite shape

 \choice definite volume

 \choice indefinite volume
\end{choices}

\question[4] How many electrons appear in the Lewis dot symbol for an
element whose electron configuration is $1s^22s^22p^3$?

\begin{choices}
\choice 2

\choice 3

\choice 5

\choice 7

\choice 8
\end{choices}


\question[5]
If the density of a solution is $3.2 \frac{grams}{mL}$, how many mL would it take to get 100 grams?

\question[10]
Fill in the following table of names and symbols of elements:

\begin{tabular}{| l | l |}
\hline
Symbol & Name \\
\hline
Au & \\
Pb & \\
 & Sodium\\
 & Hydrogen\\
Hg & \\
Li & \\
Be & \\
 & Nitrogen \\
 & Chlorine \\
P & \\
\hline
\end{tabular}

\question[10]
Fill in the following table:

\begin{tabular}{| c | c | c | c | c | c |}
\hline
Symbol & Atomic & Mass & Number of&Number of&Number of \\
 & Number & Number & Protons & Neutrons & Electrons \\
\hline
$^{23}\textbf{V}$ & & & & & \\
\hline
 & 55 & 133 & & & 55\\
\hline
& & 77 & 33 & & 33\\ 
\hline
$\textbf{Br}^{1-}$ & & & & & \\
\hline
& & & 20 & 20 & 18 \\ 
\hline
\end{tabular}

\question[10]  Write chemical formulas for the compounds formed between the following positive and negative ions.
\begin{parts}
\part  $\textbf{Al}^{3+}$ and $\textbf{CO}_3^{2-}$ 

\part $\textbf{Fe}^{3+} $ and $\textbf{OH}^-$  
\end{parts}


\question[10]
The following Lewis diagrams are for two unspecified elements shown in
their \emph{neutral form}.  How many atoms of each type would be
required to form a neutral \emph{compound}?  What is the charge on
each ion type? 

\begin{parts}
\part \lewis{X}{.}{.}{.}{.}{.}{.}{.}{}  - \lewis{Y}{}{}{.}{}{}{}{}{}

\part \lewis{X}{.}{.}{.}{.}{}{}{.}{.} - \lewis{Y}{.}{}{.}{}{.}{}{}{}
\end{parts}


\question[10]  Fill in a shell diagram for a Phosphorus (\textbf{P}) atom.
  \begin{parts}
    \part How many shells (n-levels) does it need?

    \part How many electrons fit in each of the inner (core) shells?

    \part How many electrons are in the outer (valence) shell?

    \part Fill in the diagram.

    \part If \textbf{P} had a full outer shell, what would the overall charge be?
 \end{parts}

\question[5] Match up one entry from the left column with an entry from the other
  column that has similar chemical properties, and, asssuming that
  these represent \emph{neutral} atoms, what elements do they each represent?

\begin{tabular}{|l|l|l|l|}
a& $1s^22s^22p^6$ & 2& $1s^22s^22p^63s^2$ \\
b& $1s^22s^1$ & 1& $1s^22s^22p^63s^23p^6$ \\
c& $1s^22s^2$ & 3& $1s^22s^22p^63s^23p^3$ \\
d& $1s^22s^22p^3$ & 4&$1s^22s^22p^63s^1$  
\end{tabular}

\question[5]  Explain \emph{why} the elements in [12] have similar
chemical properties.

\section{}

\question[10] Draw the Lewis dot structures for the following atoms, then
show/describe how those individual atoms form the bonds that comprise
the molecule.  You will need to determine how many of each atom type
are required.

\begin{parts}
\part \textbf{Ca  Cl}
\vspace{1cm}
\part \textbf{B  F}
\vspace{1cm}
\end{parts}

\question[10] Draw the Lewis dot structures for the following two
molecules.  First draw dot structures for the individual atoms, then
show/describe how those individual atoms form the bonds that comprise
the molecule.
\begin{parts}
\part \ce{C2H4}
\vspace{1cm}
\part \ce{C3H8}
\vspace{1cm}
\end{parts}


\question[10] Vitamin C has the formula \ce{C6H8O6}. Calculate the number
of vitamin C molecules present in a 0.250g tablet of pure vitamin C.
\vspace{3cm}

\question[20] Balance the following chemical equations: 
\begin{parts}
\part \ce{C6H12O6 + O2 -> CO2 + H2O}
\vspace{1cm}
\part \ce{KOH + C2H4O2 ->  H2O + KC2H3O2 }
\vspace{1cm}
\part \ce{IBr + NH3 -> NH4Br + NI3}
\vspace{1cm}
\part \ce{SO2Cl2 + HI -> H2S + H2O + HCl + I2}
\vspace{1cm}
\end{parts}

\newpage

\question[10] The reaction below is started with 15 grams of \ce{NaOH} and
110 grams of \ce{H2SO4}.  The chemical equation is already balanced.

\ce{2NaOH + H2SO4 -> 2H2O + Na2SO4}
\vspace{5mm}

\begin{parts}
\part One of the reagents is in excess.  Which one, and by how much? (Compute the excess in moles and in grams)
\vspace{3cm}
\part How much \ce{Na2SO4} will be produced?  (Compute the number of moles and the number of grams.)
\vspace{3cm}
\end{parts}

\question[10] How many grams of nitrogen are present in a 0.10 g
sample of caffeine \ce{C8H10N4O2}?
\vspace{2cm}

\question[10] What is the molarity of a solution when you have 250
grams of sodium chloride \ce{NaCl} in 400 mL?
\vspace{2cm}

\question[10] What is the molarity of the solution prepared by
diluting 65mL of 0.95 M \ce{Na2SO4} solution to a final volume of 135mL?
\vspace{2cm}

\section{}

\question[15] How many hydrogen atoms are in:
\begin{parts}
\part An alkane containing 17 carbon atoms?
\part An alkene (with 5 double bonds) containing 43 carbon atoms?
\part A cycloalkane containing 10 carbon atoms?
\end{parts}

\question[25] Complete the following table

\begin{tabular}{l|l|l}
Name & Expanded Structural Formula & Chemical Formula \\
\hline
Pentane & & \\
 & & \\
 & & \\
 & & \\
\hline
Hexane & & \\
& & \\ & & \\ & & \\ \hline
2,3,4-trimethylheptane & & \\
& & \\ & & \\ & & \\ \hline
3-ethyl-4,5-dipropyloctane & & \\
& & \\ & & \\ & & \\ \hline
2-propanol
& & \\ & & \\ & & \\ \hline
propene
& & \\ & & \\ & & \\ \hline
3,5-dimethyl-1-hexene & & \\
& & \\ & & \\ & & \\ \hline
\end{tabular}


\question[5] Rank the following in order of increasing heat capacity.
Explain what properties of the molecules affect the heat capacity.

2,3,4-octotriene, octane, octene, and 2,3-octodiene
\vspace{1cm}

\question[5] Rank the following in order of increasing boiling point:
2,3,4-triethylhexane, 2-ethylhexane, hexane, 2,3-diethylhexane, benzene 
\newpage


\question[5] Describe a process you could use to separate the components
of a sample that contained equal parts of butane, hexane, methane, and
octane.
\vspace{1cm}

\question[5] What is the difference between a saturated and an
unsaturated hydrocarbon?

\vspace{1cm}

\question[20] Determine all products of the following reactions
and provide IUPAC names for both the reactants and the products.
\begin{parts}
\part

\part 
\vspace{5cm}

\end{parts}
\vspace{5cm}






\question[10] A sample of gaseous cyclopropane, with a volume of 400 mL at a
temperature of 250C, is cooled at a constant pressure to 150C.  What
is the new volume, in mL, of the sample?
\vspace{2cm}


\question[10] A sample of \ce{O2} gas occupies a volume of 1.02 L at 0.9 atm
pressure and a temperature of 300C.  What volume, in L, will this gas
sample occupy at 0.8 atm pressure and 540C?




\end{questions}
\end{document}