\documentclass[addpoints, 12pt]{exam}
\pagestyle{headandfoot}
\firstpageheader{Name:}{Exam III}{June 21, 2012}
\firstpageheadrule
\pointsinmargin

\usepackage{lewis}

\begin{document}
\begin{questions}
Exam III
Name:__________________________							Spring 2kk012
Instructions:  In order to receive full credit you MUST correctly spell the appropriate IUPAC names.
1. How many hydrogen atoms are in:
a. An alkane containing 17 carbon atoms:________________
b. An alkene containing 43 carbon atoms:_________________
2. Complete the following table
Name
Draw (choose line angle or skeletal)
Pentane

2,3,4-trimethylheptane



3-ethyl-4,5-dipropyloctane





ethenylcyclohexane

propene

4-tert-butyloctane

Trans-1,2-dimethylcyclopentane





3,5-dimethyl-1-hexene

2-propanol



3. Draw all of the possible isomers for a 5 carbon alkane.



4. Explain the difference between an isomer and a conformational structure.


5.  What is wrong with this name: if there is nothing wrong report “no change” or if it’s wrong rewrite it:
a.  2-3, dimethyl-4,ethyl-5 propylnonane ___________________________________________________
b. 5-sec-butyl-4-isopropyl-3-methyldecane _________________________________________________
6. What is the general name of an unsaturated hydrocarbon that contains one or more carbon-carbon triple bonds?

7. Using the Henderson-Hasselbach equation (pH=pka + log [A-]/[HA]): What is the pH of a buffer solution that is 0.5M in formic acid (HCHO2) and 1.0M in sodium formate (NaCHO2)? The pKa for formic acid 3.74. 




8. Short answer
a. Predict what would happen if you add a base to a buffered system.

b. Predict what would happen if you add an acid to a buffered system.


9. True or False.  Alcohols cannot participate in hydrogen bonding.


10. Complete the following table of chemical reactions
Reactants (NOTE: All reactants)
Conditions
Products (NOTE: All products) 
H2C=CH2 + H-OH 
H2SO4

        O
          ||
H3C-C-CH3 + H2

Catalyst

CH3CH2CHCH2CH2CH3
             |
             OH
H2SO4 @ 180OC


H2SO4 @ 140OC
CH3CH2CH2OCH2CH2CH3 + H-OH
CH3CH2OH
KMnO4


KMnO4
        O
          ||
H3C-C-CH3
11.  Draw the general molecular structures representative of primary, secondary, and tertiary alcohols.


12. Can a tertiary alcohol undergo oxidation? Why or why not (explain).

13. Using Zaitsev’s rule predict the predominant product from the intramolecular alcohol dehydration reaction and explain why you chose this as the predominant product: 




Bonus: In regards to the table in question #10, name the organic molecules that you drew in rows number 1 ,2 ,4 and 6.
1.________________________________________________________________________
2.________________________________________________________________________
4.________________________________________________________________________
6.________________________________________________________________________ 




\end{questions}

\end{document}
