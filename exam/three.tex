\documentclass[addpoints, 12pt]{exam}
\pagestyle{headandfoot}
\firstpageheader{Name:}{Exam III}{June 21, 2012}
\firstpageheadrule
\pointsinmargin

\usepackage{lewis}
\usepackage[version=3]{mhchem}

\begin{document}
\begin{questions}

\question[15] How many hydrogen atoms are in:
\begin{parts}
\part An alkane containing 17 carbon atoms?
\part An alkene (with 5 double bonds) containing 43 carbon atoms?
\part A cycloalkane containing 10 carbon atoms?
\end{parts}

\question[25] Complete the following table

\begin{tabular}{l|l|l}
Name & Expanded Structural Formula & Chemical Formula \\
\hline
Pentane & & \\
 & & \\
 & & \\
 & & \\
\hline
Hexane & & \\
& & \\ & & \\ & & \\ \hline
2,3,4-trimethylheptane & & \\
& & \\ & & \\ & & \\ \hline
3-ethyl-4,5-dipropyloctane & & \\
& & \\ & & \\ & & \\ \hline
2-propanol
& & \\ & & \\ & & \\ \hline
propene
& & \\ & & \\ & & \\ \hline
3,5-dimethyl-1-hexene & & \\
& & \\ & & \\ & & \\ \hline
\end{tabular}


\question[5] Rank the following in order of increasing heat capacity.
Explain what properties of the molecules affect the heat capacity.

2,3,4-octotriene, octane, octene, and 2,3-octodiene
\vspace{1cm}

\question[5] Rank the following in order of increasing boiling point:
2,3,4-triethylhexane, 2-ethylhexane, hexane, 2,3-diethylhexane, benzene 
\newpage


\question[5] Describe a process you could use to separate the components
of a sample that contained equal parts of butane, hexane, methane, and
octane.
\vspace{1cm}

\question[5] What is the difference between a saturated and an
unsaturated hydrocarbon?

\vspace{1cm}

\question[20] Determine all products of the following reactions
and provide IUPAC names for both the reactants and the products.
\begin{parts}
\part

\part 
\vspace{5cm}

\end{parts}
\vspace{5cm}






\question[10] A sample of gaseous cyclopropane, with a volume of 400 mL at a
temperature of 250C, is cooled at a constant pressure to 150C.  What
is the new volume, in mL, of the sample?
\vspace{2cm}


\question[10] A sample of \ce{O2} gas occupies a volume of 1.02 L at 0.9 atm
pressure and a temperature of 300C.  What volume, in L, will this gas
sample occupy at 0.8 atm pressure and 540C?







\end{questions}

\end{document}
