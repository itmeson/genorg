\documentclass[addpoints, 12pt]{exam}
\pagestyle{headandfoot}
\firstpageheader{Name:}{Exam III}{June 21, 2012}
\firstpageheadrule
\pointsinmargin

\usepackage{lewis}
\usepackage{mhchem}

\begin{document}
\begin{questions}

\question How many hydrogen atoms are in:
\begin{parts}
\part An alkane containing 17 carbon atoms?
\part An alkene (with 5 double bonds) containing 43 carbon atoms?
\end{parts}

\question Complete the following table

\begin{tabular}{l|l|l}
Name & Expanded Structural Formula & Chemical Formula \\
\hline
Pentane & & \\
 & & \\
 & & \\
 & & \\
\hline
Hexane & & \\
& & \\ & & \\ & & \\ \hline
2,3,4-trimethylheptane & & \\
& & \\ & & \\ & & \\ \hline
3-ethyl-4,5-dipropyloctane & & \\
& & \\ & & \\ & & \\ \hline
2-propanol
& & \\ & & \\ & & \\ \hline
propene
& & \\ & & \\ & & \\ \hline
3,5-dimethyl-1-hexene & & \\
& & \\ & & \\ & & \\ \hline
\end{tabular}


\question Rank the following in order of increasing heat capacity.
Explain what properties of the molecules affect the heat capacity.

2,3,4-octotriene, octane, octene, and 2,3-octodiene


\question Rank the following in order of increasing boiling point:
\begin{parts}
\part butane, hexane, propane, methane, octane
\part 2,3,4-trimethyloctane, 2-methyloctane, octane, 2,3-dimethyloctane, 
\end{parts}

\question Describe a process you could use to separate the components
of a sample that contained equal parts of butane, hexane, methane, and
octane.

\question 




\question What is the general name of an unsaturated hydrocarbon that
contains one or more carbon-carbon triple bonds?



9. True or False.  Alcohols cannot participate in hydrogen bonding.


10. Complete the following table of chemical reactions
Reactants (NOTE: All reactants)
Conditions
Products (NOTE: All products) 
H2C=CH2 + H-OH 
H2SO4

        O
          ||
H3C-C-CH3 + H2

Catalyst

CH3CH2CHCH2CH2CH3
             |
             OH
H2SO4 @ 180OC


H2SO4 @ 140OC
CH3CH2CH2OCH2CH2CH3 + H-OH
CH3CH2OH
KMnO4


KMnO4
        O
          ||
H3C-C-CH3
11.  Draw the general molecular structures representative of primary, secondary, and tertiary alcohols.


12. Can a tertiary alcohol undergo oxidation? Why or why not (explain).

13. Using Zaitsev’s rule predict the predominant product from the intramolecular alcohol dehydration reaction and explain why you chose this as the predominant product: 






\question A sample of gaseous cyclopropane, with a volume of 400 mL at a
temperature of 250C, is cooled at a constant pressure to 150C.  What
is the new volume, in mL, of the sample?



14. A sample of O2 gas occupies a volume of 1.02 L at 755 mm Hg
pressure and a temperature of 00C.  What volume, in L, will this gas
sample occupy at 735 mm Hg pressure and 540C?


15. Calculate the volume, in L, occupied by 1.52 moles of carbon
monoxide gas at 0.992 atm pressure and a temperature of 650C.





17. Explain whether or not the possibility of hydrogen bonding between
a water molecule and an ammonia molecule can occur.



%Bonus: In regards to the table in question #10, name the organic molecules that you drew in rows number 1 ,2 ,4 and 6.
%1.________________________________________________________________________%
%2.________________________________________________________________________
%4.________________________________________________________________________
%6.________________________________________________________________________ 




\end{questions}

\end{document}
