\documentclass[addpoints, 12pt]{exam}
\pagestyle{headandfoot}
\firstpageheader{Name:}{Final Exam}{June 28, 2012}
\firstpageheadrule
\pointsinmargin

\usepackage{lewis}

\begin{document}
\begin{questions}

SCI2045					FINAL EXAM						Sp2012
Name:______________________________________________________
Part I:  Matching 
Directions: Match the term in Column A to its appropriate structure in Column B by placing the number from Column A in the “Answer (#)” Column next to its appropriate structure in Column B.  Each is worth 1 point.
Column A
Answer (#)
Column B
1.  Alkyne

C-C-SH 
2.  Amine


3.  Carboxylic Acid


4.  Acid Anhydride


5.  Acetal


6.  Aldehyde

R-O-R
7.  Alkane

C-O-C
8. Benzene


9. Cycloalkane


10. Alkene


11. Alcohol

CH3-CH2-NH2
12. Ketone


13. Ester


14. Amide


15. Acid Chloride


16. Ether


17. Cycloalkene


18. Hydroxy acid

C-S-C 
19. Unsaturated acid


20. Thial


21. Phenol


22. Thione


23. Hemiacetal


24. Ether


25. Thiol

CH4
26. Thioether


27. Keto acid


28. Thioester



Part II: Naming and Structure
Directions: Complete the chart below by either providing an IUPAC name or a structure as indicated by an empty box.  When you are to draw a structure you may choose between expanded, condensed, line angle, or skeletal view.  Spelling does count. Each is worth 2 points.
#
Name
Structure
29.
2-ethylpentanal

30.


31.
Cyclobutanone

32.


33.
Butanoic acid

34.
Benzoic acid

35.


36.


37.
N-methylaniline

38.

H2N-CH2-CH2-CH2-CH2-NH2
39.
4-amino-2-pentanone

40.
Benzamide

41.


42.
2-ethoxybutane

43.
2-methylphenol

44.



Part III:  Basic Chemical Reactions of Organic Molecules.
Directions: Fill in the chart below.  Provide either the reactants or products of each chemical reaction as indicated by an empty box or “?”. Points are indicated for each problem. Note: [O]= Mild oxidizing agents; H+= acid catalyst
#
Reactants
Conditions
Products
Points
45

[O]

1
46

[O]

1
47
 + H2
Ni

1
48


?                                                 + H2
Ni

1
49

H+
Hemiacetal
1
50
Hemiacetal + alcohol
H+

1

+ H-OH
H+
?+?+?
3
51
 + NaOH

? + H2O
1
52
+ H-OH
H+
?+ ?
2
53
+


1
54
10 amine + alkyl halide
base

1
55
+ R-NH2
1000C
Catalyst
? + H2O
1

Part IV: Cumulative Multiple Choice Questions.  Each worth ¼ point.
56.  The modern periodic table arranges the elements in order of
A)  year of discovery.
B)  decreasing size of the nucleus.
C)  increasing reactivity with oxygen.
D)  increasing number of protons.
57.  An element is a substance which
A)  can be broken down into simpler substances by physical means.
B)  cannot be broken down into simpler substances by physical means.
C)  can be broken down into simpler substances by chemical means.
D)  cannot be broken down into simpler substances by chemical means.
58.  Which of the following statements concerning subatomic particles is correct?
A)  Four fundamental types exist, all of which are charged.
B)   Four fundamental types exist, two of which are charged.
C)  Three fundamental types exist, none of which are charged.
D)  Three fundamental types exist, two of which are charged.
59.  The nucleus of an atom
A)  contains only electrons
B)  has no charge because of the presence of neutrons
C)  has no mass
D)  accounts for a large amount of mass of an atom
60.  Which of the following is a property of both gases and liquids?
A)  definite shape
B)  indefinite shape
C)  definite volume
D)  indefinite volume
61.  The “octet rule” relates to the number eight because
A)  only atoms with 8 valence electrons undergo chemical reaction.
B)  ions with charges of +8 and –8 are very stable.
C)  atoms, during compound formation, frequently obtain 8 valence electrons.
D)  all electron subshells can hold 8 electrons.
62.  Which of the following elements would have a Lewis symbol that contains 5 electrons?
A)	nitrogen
B)	fluorine
C)	phosphorus
D)	more than one correct response
E)	no correct response
63.  Which of the following statements contrasting covalent bonds and  ionic bonds is correct?
A)	Covalent bonds usually involve two nonmetals and ionic bonds usually involve two metals.
B)	Covalent bonds usually involve two metals and ionic bonds usually involve a metal and a nonmetal.
C)	Covalent bonds usually involve a metal and a nonmetal and ionic bonds usually involve two nonmetals.
D)	Covalent bonds usually involve two nonmetals and ionic bonds usually involve a metal and a nonmetal.
64.  Electronegativity is a concept that is useful along with other concepts in
A)	predicting the polarity of a bond.
B)	deciding how many electrons are involved in a bond.
C)	formulating a statement of the octet rule.
D)	determining the charge on a polyatomic ion.
65.  To determine the formula mass of a compound you should
A)	add up the atomic masses of all the atoms present.
B)	add up the atomic masses of all the atoms present and divide by the number of atoms present.
C)	add up the atomic numbers of all the atoms present.
D)	add up the atomic numbers of all the atoms present and divide by the number of atoms present.



66.  A mole of a chemical substance represents
A)	the mass of the substance that will combine with 12.0 g of carbon.
B)	the mass of the substance that will combine with 100.0 g of oxygen.
C)	6.02 x 1023 chemical particles of the substance.
D)	6.02 x 10–23 grams of the substance.
67.  Which of the following equations is balanced?
A)	FeS + 2HBr  FeBr2 + 2H2S
B)	PbO2 + 2H2  Pb + 2H2O
C)	Fe2O3 + 3H2  2Fe + 3H2O
D)	more than one correct response
E)	no correct response
68.  In a solution, the solvent is
A)	the substance being dissolved.
B)	always a liquid.
C)	the substance present in the greatest amount.
D)	always water.
69.  The defining expression for the molarity concentration unit is
A)	moles of solute/liters of solution.
B)	moles of solute/liters of solvent.
C)	grams of solute/liters of solution.
D)	grams of solute/liters of solvent.
70.  Which of the following general reaction types is characterized by there being a single reactant?
A)	combination
B)	decomposition
C)	single-replacement
D)	double-replacement
71,  Which of the following reactions is a nonredox decomposition reaction?
A)	2CuO  2Cu + O2
B)	2KClO3  2KCl + 3 O2
C)	CaCO3  CaO + CO2
D)	SO2 + H2O  H2SO3




72.  For a system at chemical equilibrium,
A)	forward and reverse reaction rates are equal.
B)	forward and reverse reaction rates are zero.
C)	reactant and product concentrations are equal.
D)	reactant and product concentrations are zero.
73.  Catalysts are correctly characterized by each of the following statements except one.  The exception is
A)	they can be either solids, liquids or gases.
B)	they lower the activation energy for a reaction.
C)	they do not actively participate in a reaction.
D)	they are not consumed in a reaction.
74.  According to the Bronsted-Lowry theory, a base is a(n)
A)	proton donor.
B)	proton acceptor.
C)	electron donor.
D)	electron acceptor.
75.  Which of the following does not describe an acidic solution?
A)	The pH is less than 7.0
B)	The [OH–] is greater than 1.0 x 10–7
C)	The [H3O+] is greater than the [OH–]
D)	The [OH–] is 6.0x 10–10
76.  A solution with a pH of 12.3 is
A)	strongly acidic
B)	weakly acidic
C)	weakly basic
D)	strongly basic
77.  If a solution is buffered at a pH of 9.8, then the addition of a small amount of acid will cause the solution to have a pH of approximately
A)	7.8
B)	8.8
C)	9.8
D)	10.8


BONUS:
Provide the IUPAC name of the ORGANIC MOLECUES  in the grey shaded boxes in PartI and  III (9 total).


\end{questions}

\end{document}
