\documentclass[addpoints, 12pt]{exam}
\pagestyle{headandfoot}
\firstpageheader{Name:}{Final Exam}{June 28, 2012}
\firstpageheadrule
\pointsinmargin

\usepackage{lewis}
\usepackage{mhchem}

\begin{document}
\begin{questions}

\section*{Part I}

\question[32] Write the following representations for each compound
(Expanded structural formula   and   Chemical formula)

\begin{parts}
\part 2,2 dichlorobutanal
\vspace{1cm}
\part 2-methyl-3-pentanone
\vspace{1cm}
\part  hexanoic acid
\vspace{1cm}
\part  Ethyl 2-propyl butanoate
\vspace{1cm}
\part 3-methyl-1-butanamine
\vspace{1cm}
\part 4-Amino-2-pentanone
\vspace{1cm}
\part 3-methylbutanamide
\vspace{1cm}
\part Amino ethanoic acid
\end{parts}
\vspace{1cm}

\question[10] What is the difference between a saturated and an
unsaturated fat?  Speculate (write a short paragraph) as to why
saturated fats might not be as good for your diet as unsaturated fats.
\vspace{2cm}

\question[10] Two copies of the molecule 2Amino propanoic acid, (also
known as Alanine) undergo a dehydration reaction to form a simple
protein chain.  Draw expanded structural formulas for the reactants
and products.
\vspace{2cm}

\question[10] How many hydrogen bonds could possibly be formed between
a single dimethylamine molecule and
\begin{parts}
\part Other dimethylamine molecules?
\part Water molecules?
\end{parts}
Show your work by drawing the molecules and hydrogen bonds.
\vspace{2cm}


\question[10] Each of the following families of compounds contain a
carbonyl group in some fashion.  Show the general structural
representation for each family.

\begin{parts}
\part aldehyde
\part ketone
\part carboxylic acid
\part ester
\part amide
\end{parts}

\newpage


\section*{Part II}

\question[2] The modern periodic table arranges the elements in order of
\begin{choices}
\choice year of discovery.
\choice decreasing size of the nucleus.
\choice increasing reactivity with oxygen.
\choice ncreasing number of protons.
\end{choices}


\question[2]  Which of the following statements contrasting covalent bonds and  ionic bonds is correct?
\begin{choices}
\choice Covalent bonds usually involve two nonmetals and ionic bonds usually involve two metals.
\choice Covalent bonds usually involve two metals and ionic bonds usually involve a metal and a nonmetal.
\choice Covalent bonds usually involve a metal and a nonmetal and ionic bonds usually involve two nonmetals.
\choice Covalent bonds usually involve two nonmetals and ionic bonds usually involve a metal and a nonmetal.
\end{choices}

\question[2] Electronegativity is a concept that is useful along with other concepts in
\begin{choices}
\choice predicting the polarity of a bond.
\choice deciding how many electrons are involved in a bond.
\choice formulating a statement of the octet rule.
\choice determining the charge on a polyatomic ion.
\end{choices}

\question[2]To determine the formula mass of a compound you should
\begin{choices}
\choice add up the atomic masses of all the atoms present.
\choice add up the atomic masses of all the atoms present and divide by the number of atoms present.
\choice add up the atomic numbers of all the atoms present.
\choice add up the atomic numbers of all the atoms present and divide by the number of atoms present.
\end{choices}

\question[2]  The defining expression for the molarity concentration unit is
\begin{choices}
\choice moles of solute/liters of solution.
\choice moles of solute/liters of solvent.
\choice grams of solute/liters of solution.
\choice grams of solute/liters of solvent.
\end{choices}

\question[8]
Classify each of the following changes as a physical property,
chemical property, physical change, or chemical change:
\begin{parts}
  \part A substance is reacted with chlorine gas
 
  \part A substance will explode if exposed to light

  \part A substance has a high density

  \part A substance has a freezing point of $-20^{\circ} \textbf{C}$

  \part A substance is melted
\end{parts}


\question[8] Identify the elements and the number of atoms of each type for
  the following formulas:
\begin{parts}
  \part  $\textbf{Fe}_2\textbf{S}_3$

  \part  $\textbf{CaCl}_2$

  \part $\textbf{H}_2\textbf{SO}_4$

  \part $\textbf{KF} $
\end{parts}


\question[4] Choose all that apply:  An element is a substance which
\begin{choices}
  \choice can be broken down into simpler substances by physical means

  \choice cannot be broken down into simpler substances by physical means

  \choice   can be broken down into simpler substances by chemical means

  \choice  cannot be broken down into simpler substances by chemcal means
  
\end{choices}

\question[4] Which of the following is a property of both gases and
liquids?
\begin{choices}
  \choice definite shape
 
 \choice indefinite shape

 \choice definite volume

 \choice indefinite volume
\end{choices}

\question[4] How many electrons appear in the Lewis dot symbol for an
element whose electron configuration is $1s^22s^22p^3$?

\begin{choices}
\choice 2

\choice 3

\choice 5

\choice 7

\choice 8
\end{choices}


\question[5]
If the density of a solution is $4.7 \frac{grams}{mL}$, how many mL would it take to get 200 grams?
\vspace{1cm}

\question[10]
Fill in the following table:

\begin{tabular}{| c | c | c | c | c | c |}
\hline
Symbol & Atomic & Mass & Number of&Number of&Number of \\
 & Number & Number & Protons & Neutrons & Electrons \\
\hline
 & 55 & 133 & & & 55\\
\hline
& & 77 & 33 & & 33\\ 
\hline
\end{tabular}
\vspace{1cm}

\question[10]  Write chemical formulas for the compounds formed between the following positive and negative ions.
\begin{parts}
\part  $\textbf{Al}^{3+}$ and $\textbf{CO}_3^{2-}$ 

\part $\textbf{Fe}^{3+} $ and $\textbf{OH}^-$  
\end{parts}
\vspace{1cm}

\question[10]
The following Lewis diagrams are for two unspecified elements shown in
their \emph{neutral form}.  How many atoms of each type would be
required to form a neutral \emph{compound}?  What is the charge on
each ion type? 

\begin{parts}
\part \lewis{X}{.}{.}{.}{.}{.}{.}{.}{}  - \lewis{Y}{}{}{.}{}{}{}{}{}

\part \lewis{X}{.}{.}{.}{.}{}{}{.}{.} - \lewis{Y}{.}{}{.}{}{.}{}{}{}
\end{parts}
\vspace{1cm}

\question[10]  Fill in a shell diagram for a Phosphorus (\textbf{P}) atom.
  \begin{parts}
    \part How many shells (n-levels) does it need?

    \part How many electrons fit in each of the inner (core) shells?

    \part How many electrons are in the outer (valence) shell?

    \part Fill in the diagram.

    \part If \textbf{P} had a full outer shell, what would the overall charge be?
 \end{parts}
\vspace{1cm}

\question[5]\label{orbitals} Match up one entry from the left column with an entry from the other
  column that has similar chemical properties, and, asssuming that
  these represent \emph{neutral} atoms, what elements do they each represent?

\begin{tabular}{|l|l|l|l|}
a& $1s^22s^22p^6$ & 1& $1s^22s^22p^63s^2$ \\
b& $1s^22s^1$ & 2& $1s^22s^22p^63s^23p^6$ \\
c& $1s^22s^2$ & 3& $1s^22s^22p^63s^23p^3$ \\
d& $1s^22s^22p^3$ & 4&$1s^22s^22p^63s^1$  
\end{tabular}

\question[5]  Explain \emph{why} the elements in [\ref{orbitals}] have similar
chemical properties.

\vspace{2cm}

\question[10] Draw the Lewis dot structures for the following atoms, then
show/describe how those individual atoms form the bonds that comprise
the molecule.  You will need to determine how many of each atom type
are required.
\textbf{Ca  Cl}
\vspace{1cm}

\question[10] Draw the Lewis dot structures for the following two
molecules.  First draw dot structures for the individual atoms, then
show/describe how those individual atoms form the bonds that comprise
the molecule.
\begin{parts}
\part \ce{C2H4}
\vspace{1cm}
\part \ce{C3H8}
\vspace{1cm}
\end{parts}


\question[12] Caffeine has the formula \ce{C8H10N4O2}. In a 0.400g
sample of pure caffeine, there are
\begin{parts}
\part How many molecules of caffeine?
\part How many atoms of nitrogen?
\part How many grams of nitrogen?
\end{parts}
\vspace{3cm}

\question[10] Balance the following chemical equations: 
\begin{parts}
\part \ce{C6H12O6 + O2 -> CO2 + H2O}
\vspace{1cm}
\part \ce{SO2Cl2 + HI -> H2S + H2O + HCl + I2}
\vspace{1cm}
\end{parts}

\newpage
\question[10] The reaction below is started with 25 grams of \ce{NaOH} and
73 grams of \ce{H2SO4}.  The chemical equation is already balanced.

\ce{2NaOH + H2SO4 -> 2H2O + Na2SO4}
\vspace{5mm}

\begin{parts}
\part One of the reagents is in excess.  Which one, and by how much? (Compute the excess in moles and in grams)
\vspace{3cm}
\part How much \ce{Na2SO4} will be produced?  (Compute the number of moles and the number of grams.)
\vspace{3cm}
\end{parts}

\question[10] How many grams of nitrogen are present in a 7 gram
sample of aminoethanoic acid \ce{H2NCCOOH}?
\vspace{2cm}

\question[10] What is the molarity of a solution prepared by
dissolving 500 grams of sodium hydroxide \ce{NaOH} in 500 mL of water?
\vspace{2cm}


\question[12] How many hydrogen atoms are in:
\begin{parts}
\part An alkane containing 12 carbon atoms?
\part An alkene (with 7 double bonds) containing 23 carbon atoms?
\part A cycloalkane containing 15 carbon atoms?
\part An alkyne (with 2 triple bonds) containing 11 carbon atoms?
\end{parts}

\newpage
\question[25] Complete the following table

\begin{tabular}{l|c|c}
Name & \hspace{1.5cm} Expanded Structural Formula     \hspace{1.5cm}        & Chemical \\
& & Formula \\
\hline
Butane & & \\
 & & \\
 & & \\
 & & \\
\hline
Heptane & & \\
& & \\ & & \\ & & \\ \hline
2,3,4-trimethylnonane & & \\
& & \\ & & \\ & & \\ \hline
3-methyl & & \\
  -4,5-dibutyloctane & & \\
& & \\ & & \\ & & \\ & & \\ & & \\ & & \\ \hline
2-propanol
& & \\ & & \\ & & \\ \hline
butene
& & \\ & & \\ & & \\ \hline
3,5-dimethyl-2-hexene & & \\
& & \\ & & \\ & & \\ 
& & \\ & & \\ & & \\\hline
\end{tabular}


\question[5] Rank the following in order of increasing heat capacity.
Explain what properties of the molecules affect the heat capacity.
2,3,4-octotriene, octane, octene, and 2,3-octodiene

\vspace{1cm}

\newpage
\question[10] Rank the following in order of increasing boiling point:
\begin{parts}
\part 2,3,4-triethylhexane, 2-ethylhexane, hexane, 2,3-diethylhexane
\part propane, nonane, decane, heptane, 2-methylpropane 
\end{parts}
\vspace{1cm}


\question[20] Determine all products of the following reactions
and provide IUPAC names for both the reactants and the products.
\begin{parts}
\part 

\part 
\vspace{5cm}

\end{parts}
\vspace{5cm}




\question[10] A sample of gaseous propane, at a pressure of 5 atmospheres(ATM) at a
temperature of 300C, is heated at a constant volume to 500C.  What
is the new pressure, in atmospheres, of the sample?  What would happen
to the container the propane was held in, if you did this?
\vspace{2cm}





\end{questions}
\end{document}