\documentclass[addpoints, 12pt]{exam}
\pagestyle{headandfoot}
\firstpageheader{Name:}{Exam II}{June 14, 2012}
\firstpageheadrule
\pointsinmargin

\usepackage{lewis}
\usepackage{mhchem}

\begin{document}
\begin{questions}

\question[10] Draw the Lewis dot structures for the following atoms, then
show/describe how those individual atoms form the bonds that comprise
the molecule.  You will need to determine how many of each atom type
are required.

\begin{parts}
\part \textbf{Ca  Cl}
\vspace{1cm}
\part \textbf{B  F}
\vspace{1cm}
\end{parts}

\question[10] Draw the Lewis dot structures for the following two
molecules.  First draw dot structures for the individual atoms, then
show/describe how those individual atoms form the bonds that comprise
the molecule.
\begin{parts}
\part \ce{C2H4}
\vspace{1cm}
\part \ce{C3H8}
\vspace{1cm}
\end{parts}


\question[10] Vitamin C has the formula \ce{C6H8O6}. Calculate the number
of vitamin C molecules present in a 0.250g tablet of pure vitamin C.
\vspace{3cm}

\question[20] Balance the following chemical equations: 
\begin{parts}
\part \ce{C6H12O6 + O2 -> CO2 + H2O}
\vspace{1cm}
\part \ce{KOH + C2H4O2 ->  H2O + KC2H3O2 }
\vspace{1cm}
\part \ce{IBr + NH3 -> NH4Br + NI3}
\vspace{1cm}
\part \ce{SO2Cl2 + HI -> H2S + H2O + HCl + I2}
\vspace{1cm}
\end{parts}

\newpage

\question[10] The reaction below is started with 15 grams of \ce{NaOH} and
110 grams of \ce{H2SO4}.  The chemical equation is already balanced.

\ce{2NaOH + H2SO4 -> 2H2O + Na2SO4}
\vspace{5mm}

\begin{parts}
\part One of the reagents is in excess.  Which one, and by how much? (Compute the excess in moles and in grams)
\vspace{3cm}
\part How much \ce{Na2SO4} will be produced?  (Compute the number of moles and the number of grams.)
\vspace{3cm}
\end{parts}

\question[10] How many grams of nitrogen are present in a 0.10 g
sample of caffeine \ce{C8H10N4O2}?
\vspace{2cm}

\question[10] What is the molarity of a solution when you have 250
grams of sodium chloride \ce{NaCl} in 400 mL?
\vspace{2cm}

\question[10] What is the molarity of the solution prepared by
diluting 65mL of 0.95 M \ce{Na2SO4} solution to a final volume of 135mL?
\vspace{2cm}

\newpage
\question[10] (Bonus) A homeopathic solution at 10C has been diluted
by a factor of $10^{20}$.  If you started with a well-mixed 1.0 M
sodium chloride \ce{NaCl} solution, and diluted it by a 10C factor,
how many sodium atoms would be left in a 1 Liter sample of the
solution?  What C factor would be required to expect that NO atoms of
sodium will be left?
\vspace{2cm}

\end{questions}

\end{document}
