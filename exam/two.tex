\documentclass[addpoints, 12pt]{exam}
\pagestyle{headandfoot}
\firstpageheader{Name:}{Exam II}{June 14, 2012}
\firstpageheadrule
\pointsinmargin

\usepackage{lewis}

\begin{document}
\begin{questions}


\question[10]  Match the chemical reaction in column A to its appropriate reaction type in column B.

\begin{tabular}{|l|l|l|}
Answer & Column A & Column B\\
\hline
A. & $N_2 + 3H \rightarrow 2NH_3$ &  1. Combustion\\
B. & $2CuO \rightarrow 2Cu + O_2$ & 2. Single Replacement \\
C. & $Fe + CuSO_4 \rightarrow Cu + FeSO_4$ & 3. Double Replacement\\
D. & $AgNO_3(aq) + NaCl(aq) \rightarrow AgCl(s) + NaNO_3(aq)$ & 4. Decomposition \\
E. & $CS_2 + 3O_2 \rightarrow CO_2 + 2SO_2$ & 5. Combination\\
\hline
\end{tabular}

2. Draw the Lewis structure using V.S.E.P.R. criteria for the simplest binary compounds that can be formed from the following pairs of nonmetals:  Sulfur and Hydrogen

3. What is the name of the covalent molecule that you made in question \#2?

4. What is the molecular geometry of the molecule that you constructed in question\#2?

5. Arrange the following elements in INCREASING order of electronegativity: Al, I, S, O, N.

6. Using either delta notation or the “plus sign arrow” notation, describe, by placing the notations around the molecule below, the bond polarity that exists for hydrochloric acid:  

H-Cl

7. Is the molecule in question \#6 polar or non-polar? And explain.


8. Classify each bond type below as either: polar covalent, nonpolar covalent, or ionic:
%a. N-Cl; electronegativities: N= 3.0 and Cl= 3.0___________________________
%b. Ca-F; electronegativities: Ca= 4.0 and F= 1.0__________________________
%c. C-O; electronegativities: C= 3.5 and O= 2.5___________________________

9. Vitamin C has the formula C6H8O6. Calculate the number of vitamin C molecules present in a 0.250g tablet of pure vitamin C.



10. Balance the following chemical equation: C6H12O6 + O2  CO2 + H2O


11. How many grams of nitrogen are present in a 0.10 g sample of caffeine (C8H10N4O2)?


12. A sample of O2 gas occupies a volume of 3.50 L at a pressure of 740 mm Hg and a temperature of 250C.  What volume will it occupy, in liters, if the pressure is increased to 790 mm Hg with no change in temperature?

13. A sample of gaseous cyclopropane, with a volume of 400 mL at a temperature of 250C, is cooled at a constant pressure to 150C.  What is the new volume, in mL, of the sample?



14. A sample of O2 gas occupies a volume of 1.02 L at 755 mm Hg pressure and a temperature of 00C.  What volume, in L, will this gas sample occupy at 735 mm Hg pressure and 540C?


15. Calculate the volume, in L, occupied by 1.52 moles of carbon monoxide gas at 0.992 atm pressure and a temperature of 650C.


16. The total pressure exerted by a mixture of the three gases oxygen, nitrogen, and water vapor is 702 mm Hg.  The partial pressures of the nitrogen and oxygen in the sample are 516 mm Hg and 123 mm Hg, respectively.  What is the partial pressure of the water vapor present in the mixture?



17. Explain whether or not the possibility of hydrogen bonding between a water molecule and an ammonia molecule can occur.


18. Calculate the mass percent when 2.13g of AgNO3 dissolved in 30.0g of H2O.


19. What is the percent by volume of acetone in an aqueous solution made by diluting 75 mL of pure acetone with water to give a volume of 785 mL of solution?  



20. What is the molarity of a solution when you have 250 grams of sodium chloride in 400 mL?



21. What is the osmolarity of a 3M sucrose solution?

22. What is the molarity of the solution prepared by diluting 65mL of 0.95 M Na2SO4 solution to a final volume of 135mL?


23. If an 8 M salt solution is separated from a 7 M salt solution by a semipermeable membrane, to which solution will water move towards?

24. Is the following reaction a redox or a nonredox reaction:  CaCO3  CaO + CO2.   Make sure to show all of your calculations.


25. For the redox reaction : FeO + CO  Fe + CO2  identify the following: the substances oxidized, reduced, the oxidizing agent, and the reducing agent.

26. Calculate the value of the equilibrium constant for the equilibrium system:                  C(s) + H2O(g) <= > CO(g) + H2(g).  At 10000C, given that the equilibrium concentrations are 0.0026M for C, 0.084M for H2, 0.06M for CO, and 0.0042M for H2O.



27. Selected information about four solutions, each at 240C, is given in the following table.  Complete the table by filling in the missing information.
[H3O+]
[OH-]
pH
Acidic or Basic


8.73


7.2x10-5


6.3x10-3





2.00




28. Acid A has a Ka value of 5.71, and acid B has a Ka value of 5.30.  Which of the two acids is the weaker acid?   

29. Calculate the pKA value of phosphoric acid (H3PO4), Ka= 7.5x10-3.


30. Write equations showing all steps in the complete dissociation of arsenic acid (H3AsO4) in water.




31. Complete the following Acid-Base Neutralization reaction by filling in the product(s):  HCl + NaOH %____________________________________  ?  And place the molecules from this reaction to its appropriate label below.
%a. Bronsted-Lowry Acid	__________________________________%
%b. B-L Base			__________________________________
%c. B-L Conjugate Acid	__________________________________
%d. B-L Conjugate Base	__________________________________

32. A 0.0500 M solution of a base is a 7.5% ionized.  Calculate the base ionization constant Kb. Hint: it would be helpful to produce a bar graph of this word problem.


\end{questions}

\end{document}
